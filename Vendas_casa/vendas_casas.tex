% Options for packages loaded elsewhere
\PassOptionsToPackage{unicode}{hyperref}
\PassOptionsToPackage{hyphens}{url}
%
\documentclass[
]{article}
\usepackage{lmodern}
\usepackage{amssymb,amsmath}
\usepackage{ifxetex,ifluatex}
\ifnum 0\ifxetex 1\fi\ifluatex 1\fi=0 % if pdftex
  \usepackage[T1]{fontenc}
  \usepackage[utf8]{inputenc}
  \usepackage{textcomp} % provide euro and other symbols
\else % if luatex or xetex
  \usepackage{unicode-math}
  \defaultfontfeatures{Scale=MatchLowercase}
  \defaultfontfeatures[\rmfamily]{Ligatures=TeX,Scale=1}
\fi
% Use upquote if available, for straight quotes in verbatim environments
\IfFileExists{upquote.sty}{\usepackage{upquote}}{}
\IfFileExists{microtype.sty}{% use microtype if available
  \usepackage[]{microtype}
  \UseMicrotypeSet[protrusion]{basicmath} % disable protrusion for tt fonts
}{}
\makeatletter
\@ifundefined{KOMAClassName}{% if non-KOMA class
  \IfFileExists{parskip.sty}{%
    \usepackage{parskip}
  }{% else
    \setlength{\parindent}{0pt}
    \setlength{\parskip}{6pt plus 2pt minus 1pt}}
}{% if KOMA class
  \KOMAoptions{parskip=half}}
\makeatother
\usepackage{xcolor}
\IfFileExists{xurl.sty}{\usepackage{xurl}}{} % add URL line breaks if available
\IfFileExists{bookmark.sty}{\usepackage{bookmark}}{\usepackage{hyperref}}
\hypersetup{
  pdftitle={Aluguel de casas: fazendo predições sobre o valor de aluguel decasas.},
  hidelinks,
  pdfcreator={LaTeX via pandoc}}
\urlstyle{same} % disable monospaced font for URLs
\usepackage[margin=1in]{geometry}
\usepackage{color}
\usepackage{fancyvrb}
\newcommand{\VerbBar}{|}
\newcommand{\VERB}{\Verb[commandchars=\\\{\}]}
\DefineVerbatimEnvironment{Highlighting}{Verbatim}{commandchars=\\\{\}}
% Add ',fontsize=\small' for more characters per line
\usepackage{framed}
\definecolor{shadecolor}{RGB}{248,248,248}
\newenvironment{Shaded}{\begin{snugshade}}{\end{snugshade}}
\newcommand{\AlertTok}[1]{\textcolor[rgb]{0.94,0.16,0.16}{#1}}
\newcommand{\AnnotationTok}[1]{\textcolor[rgb]{0.56,0.35,0.01}{\textbf{\textit{#1}}}}
\newcommand{\AttributeTok}[1]{\textcolor[rgb]{0.77,0.63,0.00}{#1}}
\newcommand{\BaseNTok}[1]{\textcolor[rgb]{0.00,0.00,0.81}{#1}}
\newcommand{\BuiltInTok}[1]{#1}
\newcommand{\CharTok}[1]{\textcolor[rgb]{0.31,0.60,0.02}{#1}}
\newcommand{\CommentTok}[1]{\textcolor[rgb]{0.56,0.35,0.01}{\textit{#1}}}
\newcommand{\CommentVarTok}[1]{\textcolor[rgb]{0.56,0.35,0.01}{\textbf{\textit{#1}}}}
\newcommand{\ConstantTok}[1]{\textcolor[rgb]{0.00,0.00,0.00}{#1}}
\newcommand{\ControlFlowTok}[1]{\textcolor[rgb]{0.13,0.29,0.53}{\textbf{#1}}}
\newcommand{\DataTypeTok}[1]{\textcolor[rgb]{0.13,0.29,0.53}{#1}}
\newcommand{\DecValTok}[1]{\textcolor[rgb]{0.00,0.00,0.81}{#1}}
\newcommand{\DocumentationTok}[1]{\textcolor[rgb]{0.56,0.35,0.01}{\textbf{\textit{#1}}}}
\newcommand{\ErrorTok}[1]{\textcolor[rgb]{0.64,0.00,0.00}{\textbf{#1}}}
\newcommand{\ExtensionTok}[1]{#1}
\newcommand{\FloatTok}[1]{\textcolor[rgb]{0.00,0.00,0.81}{#1}}
\newcommand{\FunctionTok}[1]{\textcolor[rgb]{0.00,0.00,0.00}{#1}}
\newcommand{\ImportTok}[1]{#1}
\newcommand{\InformationTok}[1]{\textcolor[rgb]{0.56,0.35,0.01}{\textbf{\textit{#1}}}}
\newcommand{\KeywordTok}[1]{\textcolor[rgb]{0.13,0.29,0.53}{\textbf{#1}}}
\newcommand{\NormalTok}[1]{#1}
\newcommand{\OperatorTok}[1]{\textcolor[rgb]{0.81,0.36,0.00}{\textbf{#1}}}
\newcommand{\OtherTok}[1]{\textcolor[rgb]{0.56,0.35,0.01}{#1}}
\newcommand{\PreprocessorTok}[1]{\textcolor[rgb]{0.56,0.35,0.01}{\textit{#1}}}
\newcommand{\RegionMarkerTok}[1]{#1}
\newcommand{\SpecialCharTok}[1]{\textcolor[rgb]{0.00,0.00,0.00}{#1}}
\newcommand{\SpecialStringTok}[1]{\textcolor[rgb]{0.31,0.60,0.02}{#1}}
\newcommand{\StringTok}[1]{\textcolor[rgb]{0.31,0.60,0.02}{#1}}
\newcommand{\VariableTok}[1]{\textcolor[rgb]{0.00,0.00,0.00}{#1}}
\newcommand{\VerbatimStringTok}[1]{\textcolor[rgb]{0.31,0.60,0.02}{#1}}
\newcommand{\WarningTok}[1]{\textcolor[rgb]{0.56,0.35,0.01}{\textbf{\textit{#1}}}}
\usepackage{graphicx}
\makeatletter
\def\maxwidth{\ifdim\Gin@nat@width>\linewidth\linewidth\else\Gin@nat@width\fi}
\def\maxheight{\ifdim\Gin@nat@height>\textheight\textheight\else\Gin@nat@height\fi}
\makeatother
% Scale images if necessary, so that they will not overflow the page
% margins by default, and it is still possible to overwrite the defaults
% using explicit options in \includegraphics[width, height, ...]{}
\setkeys{Gin}{width=\maxwidth,height=\maxheight,keepaspectratio}
% Set default figure placement to htbp
\makeatletter
\def\fps@figure{htbp}
\makeatother
\setlength{\emergencystretch}{3em} % prevent overfull lines
\providecommand{\tightlist}{%
  \setlength{\itemsep}{0pt}\setlength{\parskip}{0pt}}
\setcounter{secnumdepth}{-\maxdimen} % remove section numbering

\title{Aluguel de casas: fazendo predições sobre o valor de aluguel
decasas.}
\author{}
\date{\vspace{-2.5em}}

\begin{document}
\maketitle

\hypertarget{importauxe7uxe3o-dos-dados}{%
\section{Importação dos dados}\label{importauxe7uxe3o-dos-dados}}

\begin{Shaded}
\begin{Highlighting}[]
\KeywordTok{setwd}\NormalTok{(}\StringTok{"D:}\CharTok{\textbackslash{}\textbackslash{}}\StringTok{07 04 2020}\CharTok{\textbackslash{}\textbackslash{}}\StringTok{Documents}\CharTok{\textbackslash{}\textbackslash{}}\StringTok{Edgar}\CharTok{\textbackslash{}\textbackslash{}}\StringTok{Projetos}\CharTok{\textbackslash{}\textbackslash{}}\StringTok{Data\_sciece}\CharTok{\textbackslash{}\textbackslash{}}\StringTok{DataScience\_projects}\CharTok{\textbackslash{}\textbackslash{}}\StringTok{Vendas\_casa"}\NormalTok{)}
\KeywordTok{library}\NormalTok{(}\StringTok{"tidyverse"}\NormalTok{)}
\end{Highlighting}
\end{Shaded}

\begin{verbatim}
## -- Attaching packages --------------------------------------------------------- tidyverse 1.3.0 --
\end{verbatim}

\begin{verbatim}
## v ggplot2 3.3.2     v purrr   0.3.4
## v tibble  3.0.1     v dplyr   1.0.0
## v tidyr   1.1.0     v stringr 1.4.0
## v readr   1.3.1     v forcats 0.5.0
\end{verbatim}

\begin{verbatim}
## -- Conflicts ------------------------------------------------------------ tidyverse_conflicts() --
## x dplyr::filter() masks stats::filter()
## x dplyr::lag()    masks stats::lag()
\end{verbatim}

\begin{Shaded}
\begin{Highlighting}[]
\NormalTok{dados\textless{}{-}}\StringTok{ }\KeywordTok{read.csv}\NormalTok{(}\StringTok{"houses\_to\_rent\_v2.csv"}\NormalTok{, }\DataTypeTok{header =}\NormalTok{ T)}
\NormalTok{ndados\textless{}{-}}\KeywordTok{nrow}\NormalTok{(dados) }
\KeywordTok{View}\NormalTok{(dados)}
\end{Highlighting}
\end{Shaded}

\hypertarget{limpeza-e-preparauxe7uxe3o-dos-dados}{%
\section{Limpeza e Preparação dos
dados}\label{limpeza-e-preparauxe7uxe3o-dos-dados}}

\n Organização do cabeçalho para que seja possível manusear as variáveis
com mais facilidade.

\begin{Shaded}
\begin{Highlighting}[]
\CommentTok{\# Exlcuindo os caracteres ..R.. que estão depois do nome da variavel.}
\NormalTok{cab\textless{}{-}}\KeywordTok{names}\NormalTok{(dados)}
\NormalTok{cab\textless{}{-}cab}\OperatorTok{\%\textgreater{}\%}\StringTok{ }\KeywordTok{str\_remove\_all}\NormalTok{(}\StringTok{"..R.."}\NormalTok{)}
\KeywordTok{colnames}\NormalTok{(dados)\textless{}{-}}\StringTok{ }\NormalTok{cab}
\end{Highlighting}
\end{Shaded}

\n Os dados são separados em treino e teste, sendo 80\% para treinar e
20\% para testar o modelo.

\begin{Shaded}
\begin{Highlighting}[]
\NormalTok{n\_treino\textless{}{-}}\StringTok{ }\NormalTok{(}\DecValTok{70}\OperatorTok{/}\DecValTok{100}\NormalTok{)}\OperatorTok{*}\NormalTok{ndados}
\KeywordTok{set.seed}\NormalTok{(}\DecValTok{100}\NormalTok{)}
\NormalTok{ua\textless{}{-}}\StringTok{ }\KeywordTok{sample}\NormalTok{(n\_treino)}
\NormalTok{treino\textless{}{-}}\StringTok{ }\NormalTok{dados[ua,]; }\KeywordTok{View}\NormalTok{(treino)}
\NormalTok{teste\textless{}{-}}\StringTok{ }\NormalTok{dados[}\OperatorTok{{-}}\NormalTok{ua,]; }\KeywordTok{View}\NormalTok{(teste)}
\end{Highlighting}
\end{Shaded}

\hypertarget{treinamento-do-modelo}{%
\section{Treinamento do modelo}\label{treinamento-do-modelo}}

\begin{Shaded}
\begin{Highlighting}[]
\NormalTok{model\textless{}{-}}\StringTok{ }\KeywordTok{lm}\NormalTok{(total}\OperatorTok{\textasciitilde{}}\NormalTok{.,}\DataTypeTok{data=}\NormalTok{ treino)}
\end{Highlighting}
\end{Shaded}

\hypertarget{avaliauxe7uxe3o-da-performance-do-modelo}{%
\section{Avaliação da performance do
modelo}\label{avaliauxe7uxe3o-da-performance-do-modelo}}

\n Para a avaliação da performance do modelo foi utilizado três
métricas, sendo elas erro médio absoluto, percentual médio do erro
absoluto e coeficiente de determinação (R²). Também foi avaliado a
distribuição dos erros dentro de quartis.

\begin{Shaded}
\begin{Highlighting}[]
\NormalTok{predito\textless{}{-}}\KeywordTok{predict}\NormalTok{(model,teste)}
\NormalTok{per\textless{}{-}}\StringTok{ }\NormalTok{teste }\OperatorTok{\%\textgreater{}\%}\StringTok{ }\KeywordTok{select}\NormalTok{(}\StringTok{"city"}\NormalTok{, }\StringTok{"total"}\NormalTok{)}\OperatorTok{\%\textgreater{}\%}
\StringTok{  }\KeywordTok{mutate}\NormalTok{(predito) }\OperatorTok{\%\textgreater{}\%}\StringTok{ }\KeywordTok{mutate}\NormalTok{(}\DataTypeTok{erro=}\NormalTok{ total}\OperatorTok{{-}}\NormalTok{predito )}\OperatorTok{\%\textgreater{}\%}
\StringTok{  }\KeywordTok{mutate}\NormalTok{(}\DataTypeTok{erro\_abs=} \KeywordTok{abs}\NormalTok{(erro))}\OperatorTok{\%\textgreater{}\%}\KeywordTok{mutate}\NormalTok{(}\DataTypeTok{erro\_perc=}\NormalTok{ erro}\OperatorTok{/}\NormalTok{total)}\OperatorTok{\%\textgreater{}\%}\StringTok{ }
\StringTok{  }\KeywordTok{mutate}\NormalTok{(}\DataTypeTok{erro\_percabs=} \KeywordTok{abs}\NormalTok{(erro\_perc))}
\NormalTok{per[,}\KeywordTok{c}\NormalTok{(}\DecValTok{4}\OperatorTok{:}\DecValTok{7}\NormalTok{)]\textless{}{-}}\KeywordTok{round}\NormalTok{(per[,}\KeywordTok{c}\NormalTok{(}\DecValTok{4}\OperatorTok{:}\DecValTok{7}\NormalTok{)], }\DecValTok{5}\NormalTok{)}

\CommentTok{\# Calculando o erro medio absoluto e percentual medio}
\NormalTok{erro\_medio\textless{}{-}}\StringTok{ }\KeywordTok{mean}\NormalTok{(per}\OperatorTok{$}\NormalTok{erro\_abs); erro\_medio}
\end{Highlighting}
\end{Shaded}

\begin{verbatim}
## [1] 0.7203906
\end{verbatim}

\begin{Shaded}
\begin{Highlighting}[]
\NormalTok{erro\_percmed\textless{}{-}}\StringTok{ }\KeywordTok{mean}\NormalTok{(per}\OperatorTok{$}\NormalTok{erro\_percabs); erro\_percmed}
\end{Highlighting}
\end{Shaded}

\begin{verbatim}
## [1] 0.000202048
\end{verbatim}

\begin{Shaded}
\begin{Highlighting}[]
\KeywordTok{summary}\NormalTok{(per}\OperatorTok{$}\NormalTok{erro\_percabs)}
\end{Highlighting}
\end{Shaded}

\begin{verbatim}
##      Min.   1st Qu.    Median      Mean   3rd Qu.      Max. 
## 0.0000000 0.0000300 0.0000600 0.0002021 0.0001500 0.2589400
\end{verbatim}

\n Cálculo do coeficiente de determinação (R²), ele varia entre 0 e 1 e
indica o quão bem ajustado o modelo está, quanto mais préximo de 1
melhor a performance do modelo.

\begin{Shaded}
\begin{Highlighting}[]
\CommentTok{\# Calculando o coeficiente de determinacao}
\NormalTok{resumo\textless{}{-}}\StringTok{ }\KeywordTok{summary}\NormalTok{(model)}
\NormalTok{r2\textless{}{-}}\StringTok{ }\NormalTok{resumo}\OperatorTok{$}\NormalTok{adj.r.squared}
\NormalTok{performance\textless{}{-}}\StringTok{ }\KeywordTok{data.frame}\NormalTok{(erro\_medio, erro\_percmed,r2)}
\NormalTok{performance}
\end{Highlighting}
\end{Shaded}

\begin{verbatim}
##   erro_medio erro_percmed        r2
## 1  0.7203906  0.000202048 0.9999999
\end{verbatim}

\end{document}
