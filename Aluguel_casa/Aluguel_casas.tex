% Options for packages loaded elsewhere
\PassOptionsToPackage{unicode}{hyperref}
\PassOptionsToPackage{hyphens}{url}
%
\documentclass[
]{article}
\usepackage{lmodern}
\usepackage{amssymb,amsmath}
\usepackage{ifxetex,ifluatex}
\ifnum 0\ifxetex 1\fi\ifluatex 1\fi=0 % if pdftex
  \usepackage[T1]{fontenc}
  \usepackage[utf8]{inputenc}
  \usepackage{textcomp} % provide euro and other symbols
\else % if luatex or xetex
  \usepackage{unicode-math}
  \defaultfontfeatures{Scale=MatchLowercase}
  \defaultfontfeatures[\rmfamily]{Ligatures=TeX,Scale=1}
\fi
% Use upquote if available, for straight quotes in verbatim environments
\IfFileExists{upquote.sty}{\usepackage{upquote}}{}
\IfFileExists{microtype.sty}{% use microtype if available
  \usepackage[]{microtype}
  \UseMicrotypeSet[protrusion]{basicmath} % disable protrusion for tt fonts
}{}
\makeatletter
\@ifundefined{KOMAClassName}{% if non-KOMA class
  \IfFileExists{parskip.sty}{%
    \usepackage{parskip}
  }{% else
    \setlength{\parindent}{0pt}
    \setlength{\parskip}{6pt plus 2pt minus 1pt}}
}{% if KOMA class
  \KOMAoptions{parskip=half}}
\makeatother
\usepackage{xcolor}
\IfFileExists{xurl.sty}{\usepackage{xurl}}{} % add URL line breaks if available
\IfFileExists{bookmark.sty}{\usepackage{bookmark}}{\usepackage{hyperref}}
\hypersetup{
  pdftitle={Fazendo predições sobre os valores de aluguéis residênciais.},
  hidelinks,
  pdfcreator={LaTeX via pandoc}}
\urlstyle{same} % disable monospaced font for URLs
\usepackage[margin=1in]{geometry}
\usepackage{color}
\usepackage{fancyvrb}
\newcommand{\VerbBar}{|}
\newcommand{\VERB}{\Verb[commandchars=\\\{\}]}
\DefineVerbatimEnvironment{Highlighting}{Verbatim}{commandchars=\\\{\}}
% Add ',fontsize=\small' for more characters per line
\usepackage{framed}
\definecolor{shadecolor}{RGB}{248,248,248}
\newenvironment{Shaded}{\begin{snugshade}}{\end{snugshade}}
\newcommand{\AlertTok}[1]{\textcolor[rgb]{0.94,0.16,0.16}{#1}}
\newcommand{\AnnotationTok}[1]{\textcolor[rgb]{0.56,0.35,0.01}{\textbf{\textit{#1}}}}
\newcommand{\AttributeTok}[1]{\textcolor[rgb]{0.77,0.63,0.00}{#1}}
\newcommand{\BaseNTok}[1]{\textcolor[rgb]{0.00,0.00,0.81}{#1}}
\newcommand{\BuiltInTok}[1]{#1}
\newcommand{\CharTok}[1]{\textcolor[rgb]{0.31,0.60,0.02}{#1}}
\newcommand{\CommentTok}[1]{\textcolor[rgb]{0.56,0.35,0.01}{\textit{#1}}}
\newcommand{\CommentVarTok}[1]{\textcolor[rgb]{0.56,0.35,0.01}{\textbf{\textit{#1}}}}
\newcommand{\ConstantTok}[1]{\textcolor[rgb]{0.00,0.00,0.00}{#1}}
\newcommand{\ControlFlowTok}[1]{\textcolor[rgb]{0.13,0.29,0.53}{\textbf{#1}}}
\newcommand{\DataTypeTok}[1]{\textcolor[rgb]{0.13,0.29,0.53}{#1}}
\newcommand{\DecValTok}[1]{\textcolor[rgb]{0.00,0.00,0.81}{#1}}
\newcommand{\DocumentationTok}[1]{\textcolor[rgb]{0.56,0.35,0.01}{\textbf{\textit{#1}}}}
\newcommand{\ErrorTok}[1]{\textcolor[rgb]{0.64,0.00,0.00}{\textbf{#1}}}
\newcommand{\ExtensionTok}[1]{#1}
\newcommand{\FloatTok}[1]{\textcolor[rgb]{0.00,0.00,0.81}{#1}}
\newcommand{\FunctionTok}[1]{\textcolor[rgb]{0.00,0.00,0.00}{#1}}
\newcommand{\ImportTok}[1]{#1}
\newcommand{\InformationTok}[1]{\textcolor[rgb]{0.56,0.35,0.01}{\textbf{\textit{#1}}}}
\newcommand{\KeywordTok}[1]{\textcolor[rgb]{0.13,0.29,0.53}{\textbf{#1}}}
\newcommand{\NormalTok}[1]{#1}
\newcommand{\OperatorTok}[1]{\textcolor[rgb]{0.81,0.36,0.00}{\textbf{#1}}}
\newcommand{\OtherTok}[1]{\textcolor[rgb]{0.56,0.35,0.01}{#1}}
\newcommand{\PreprocessorTok}[1]{\textcolor[rgb]{0.56,0.35,0.01}{\textit{#1}}}
\newcommand{\RegionMarkerTok}[1]{#1}
\newcommand{\SpecialCharTok}[1]{\textcolor[rgb]{0.00,0.00,0.00}{#1}}
\newcommand{\SpecialStringTok}[1]{\textcolor[rgb]{0.31,0.60,0.02}{#1}}
\newcommand{\StringTok}[1]{\textcolor[rgb]{0.31,0.60,0.02}{#1}}
\newcommand{\VariableTok}[1]{\textcolor[rgb]{0.00,0.00,0.00}{#1}}
\newcommand{\VerbatimStringTok}[1]{\textcolor[rgb]{0.31,0.60,0.02}{#1}}
\newcommand{\WarningTok}[1]{\textcolor[rgb]{0.56,0.35,0.01}{\textbf{\textit{#1}}}}
\usepackage{graphicx}
\makeatletter
\def\maxwidth{\ifdim\Gin@nat@width>\linewidth\linewidth\else\Gin@nat@width\fi}
\def\maxheight{\ifdim\Gin@nat@height>\textheight\textheight\else\Gin@nat@height\fi}
\makeatother
% Scale images if necessary, so that they will not overflow the page
% margins by default, and it is still possible to overwrite the defaults
% using explicit options in \includegraphics[width, height, ...]{}
\setkeys{Gin}{width=\maxwidth,height=\maxheight,keepaspectratio}
% Set default figure placement to htbp
\makeatletter
\def\fps@figure{htbp}
\makeatother
\setlength{\emergencystretch}{3em} % prevent overfull lines
\providecommand{\tightlist}{%
  \setlength{\itemsep}{0pt}\setlength{\parskip}{0pt}}
\setcounter{secnumdepth}{-\maxdimen} % remove section numbering

\title{Fazendo predições sobre os valores de aluguéis residênciais.}
\author{}
\date{\vspace{-2.5em}}

\begin{document}
\maketitle

O obejetivo é fazer predições a cerca do valor médio do aluguel
residêncial. O conjunto de dados possue um total de 10962 casas. Os
dados foram obtidos através do Kaggle:
\url{https://www.kaggle.com/rubenssjr/brasilian-houses-to-rent}

\hypertarget{importauxe7uxe3o-dos-dados}{%
\section{Importação dos dados}\label{importauxe7uxe3o-dos-dados}}

\begin{Shaded}
\begin{Highlighting}[]
\KeywordTok{setwd}\NormalTok{(}\StringTok{"D:}\CharTok{\textbackslash{}\textbackslash{}}\StringTok{07 04 2020}\CharTok{\textbackslash{}\textbackslash{}}\StringTok{Documents}\CharTok{\textbackslash{}\textbackslash{}}\StringTok{Edgar}\CharTok{\textbackslash{}\textbackslash{}}\StringTok{Projetos}\CharTok{\textbackslash{}\textbackslash{}}\StringTok{Data\_sciece}\CharTok{\textbackslash{}\textbackslash{}}\StringTok{DataScience\_projects}\CharTok{\textbackslash{}\textbackslash{}}\StringTok{Aluguel\_casa"}\NormalTok{)}
\KeywordTok{library}\NormalTok{(}\StringTok{"tidyverse"}\NormalTok{)}
\KeywordTok{library}\NormalTok{(}\StringTok{"rpart"}\NormalTok{)}
\KeywordTok{library}\NormalTok{(}\StringTok{"randomForest"}\NormalTok{)}
\end{Highlighting}
\end{Shaded}

\begin{verbatim}
## Warning: package 'randomForest' was built under R version 4.0.4
\end{verbatim}

\begin{Shaded}
\begin{Highlighting}[]
\NormalTok{dados\textless{}{-}}\StringTok{ }\KeywordTok{read.csv}\NormalTok{(}\StringTok{"houses\_to\_rent\_v2.csv"}\NormalTok{, }\DataTypeTok{header =}\NormalTok{ T)}
\KeywordTok{View}\NormalTok{(dados)}
\end{Highlighting}
\end{Shaded}

\hypertarget{limpeza-e-preparauxe7uxe3o-dos-dados}{%
\section{Limpeza e Preparação dos
dados}\label{limpeza-e-preparauxe7uxe3o-dos-dados}}

Organização do cabeçalho para que seja possível manusear as variáveis
com mais facilidade. E retirando linhas em que não consta a indicação do
andar.

\begin{Shaded}
\begin{Highlighting}[]
\CommentTok{\# Exlcuindo os caracteres ..R.. que estão depois do nome da variavel.}
\KeywordTok{str}\NormalTok{(dados)}
\end{Highlighting}
\end{Shaded}

\begin{verbatim}
## 'data.frame':    10692 obs. of  13 variables:
##  $ city               : chr  "São Paulo" "São Paulo" "Porto Alegre" "Porto Alegre" ...
##  $ area               : int  70 320 80 51 25 376 72 213 152 35 ...
##  $ rooms              : int  2 4 1 2 1 3 2 4 2 1 ...
##  $ bathroom           : int  1 4 1 1 1 3 1 4 2 1 ...
##  $ parking.spaces     : int  1 0 1 0 0 7 0 4 1 0 ...
##  $ floor              : chr  "7" "20" "6" "2" ...
##  $ animal             : chr  "acept" "acept" "acept" "acept" ...
##  $ furniture          : chr  "furnished" "not furnished" "not furnished" "not furnished" ...
##  $ hoa..R..           : int  2065 1200 1000 270 0 0 740 2254 1000 590 ...
##  $ rent.amount..R..   : int  3300 4960 2800 1112 800 8000 1900 3223 15000 2300 ...
##  $ property.tax..R..  : int  211 1750 0 22 25 834 85 1735 250 35 ...
##  $ fire.insurance..R..: int  42 63 41 17 11 121 25 41 191 30 ...
##  $ total..R..         : int  5618 7973 3841 1421 836 8955 2750 7253 16440 2955 ...
\end{verbatim}

\begin{Shaded}
\begin{Highlighting}[]
\NormalTok{cab\textless{}{-}}\KeywordTok{names}\NormalTok{(dados)}
\NormalTok{cab\textless{}{-}cab}\OperatorTok{\%\textgreater{}\%}\StringTok{ }\KeywordTok{str\_remove\_all}\NormalTok{(}\StringTok{"..R.."}\NormalTok{)}
\KeywordTok{colnames}\NormalTok{(dados)\textless{}{-}}\StringTok{ }\NormalTok{cab}
\NormalTok{excl\textless{}{-}}\StringTok{ }\NormalTok{dados}\OperatorTok{$}\NormalTok{floor}\OperatorTok{\%\textgreater{}\%}\KeywordTok{str\_which}\NormalTok{(}\StringTok{"{-}"}\NormalTok{)}
\NormalTok{dados\textless{}{-}}\StringTok{ }\NormalTok{dados[}\OperatorTok{{-}}\NormalTok{excl,]}
\NormalTok{ndados\textless{}{-}}\StringTok{ }\KeywordTok{nrow}\NormalTok{(dados)}
\end{Highlighting}
\end{Shaded}

Os dados são separados em treino e teste, sendo 80\% para treinar e 20\%
para testar o modelo.

\begin{Shaded}
\begin{Highlighting}[]
\NormalTok{n\_treino\textless{}{-}}\StringTok{ }\NormalTok{(}\DecValTok{70}\OperatorTok{/}\DecValTok{100}\NormalTok{)}\OperatorTok{*}\NormalTok{ndados}
\KeywordTok{set.seed}\NormalTok{(}\DecValTok{100}\NormalTok{)}
\NormalTok{ua\textless{}{-}}\StringTok{ }\KeywordTok{sample}\NormalTok{(n\_treino)}
\NormalTok{treino\textless{}{-}}\StringTok{ }\NormalTok{dados[ua,]; }\KeywordTok{View}\NormalTok{(treino)}
\NormalTok{teste\textless{}{-}}\StringTok{ }\NormalTok{dados[}\OperatorTok{{-}}\NormalTok{ua,]; }\KeywordTok{View}\NormalTok{(teste)}
\end{Highlighting}
\end{Shaded}

\hypertarget{treinamento-do-modelo}{%
\section{Treinamento do modelo}\label{treinamento-do-modelo}}

Modelo construído utilizando regressão linear múltipla.

\begin{Shaded}
\begin{Highlighting}[]
\NormalTok{model\textless{}{-}}\StringTok{ }\KeywordTok{lm}\NormalTok{(total}\OperatorTok{\textasciitilde{}}\NormalTok{.,}\DataTypeTok{data=}\NormalTok{ treino)}
\end{Highlighting}
\end{Shaded}

\hypertarget{avaliauxe7uxe3o-da-performance-do-modelo}{%
\section{Avaliação da performance do
modelo}\label{avaliauxe7uxe3o-da-performance-do-modelo}}

Para a avaliação da performance do modelo foi utilizado três métricas,
sendo elas erro médio absoluto, percentual médio do erro absoluto e
coeficiente de determinação (R²). Também foi avaliado a distribuição dos
erros dentro de quartis.

\begin{Shaded}
\begin{Highlighting}[]
\NormalTok{predito\textless{}{-}}\KeywordTok{predict}\NormalTok{(model,teste)}
\NormalTok{per\textless{}{-}}\StringTok{ }\NormalTok{teste }\OperatorTok{\%\textgreater{}\%}\StringTok{ }\KeywordTok{select}\NormalTok{(}\StringTok{"city"}\NormalTok{, }\StringTok{"total"}\NormalTok{)}\OperatorTok{\%\textgreater{}\%}
\StringTok{  }\KeywordTok{mutate}\NormalTok{(predito) }\OperatorTok{\%\textgreater{}\%}\StringTok{ }\KeywordTok{mutate}\NormalTok{(}\DataTypeTok{erro=}\NormalTok{ total}\OperatorTok{{-}}\NormalTok{predito )}\OperatorTok{\%\textgreater{}\%}
\StringTok{  }\KeywordTok{mutate}\NormalTok{(}\DataTypeTok{erro\_abs=} \KeywordTok{abs}\NormalTok{(erro))}\OperatorTok{\%\textgreater{}\%}\KeywordTok{mutate}\NormalTok{(}\DataTypeTok{erro\_perc=}\NormalTok{ erro}\OperatorTok{/}\NormalTok{total)}\OperatorTok{\%\textgreater{}\%}\StringTok{ }
\StringTok{  }\KeywordTok{mutate}\NormalTok{(}\DataTypeTok{erro\_percabs=} \KeywordTok{abs}\NormalTok{(erro\_perc))}
\NormalTok{per[,}\KeywordTok{c}\NormalTok{(}\DecValTok{4}\OperatorTok{:}\DecValTok{7}\NormalTok{)]\textless{}{-}}\KeywordTok{round}\NormalTok{(per[,}\KeywordTok{c}\NormalTok{(}\DecValTok{4}\OperatorTok{:}\DecValTok{7}\NormalTok{)], }\DecValTok{5}\NormalTok{)}

\CommentTok{\# Calculando o erro medio absoluto e percentual medio}
\NormalTok{erro\_medio\textless{}{-}}\StringTok{ }\KeywordTok{mean}\NormalTok{(per}\OperatorTok{$}\NormalTok{erro\_abs)}
\NormalTok{erro\_percmed\textless{}{-}}\StringTok{ }\KeywordTok{mean}\NormalTok{(per}\OperatorTok{$}\NormalTok{erro\_percabs)}
\KeywordTok{summary}\NormalTok{(per}\OperatorTok{$}\NormalTok{erro\_percabs)}
\end{Highlighting}
\end{Shaded}

\begin{verbatim}
##      Min.   1st Qu.    Median      Mean   3rd Qu.      Max. 
## 0.0000000 0.0000300 0.0000700 0.0002436 0.0001700 0.2589200
\end{verbatim}

Cálculo do coeficiente de determinação (R²), ele varia entre 0 e 1 e
indica o quão bem ajustado o modelo está, quanto mais préximo de 1
melhor a performance do modelo.

\begin{Shaded}
\begin{Highlighting}[]
\CommentTok{\# Calculando o coeficiente de determinacao}
\NormalTok{resumo\textless{}{-}}\StringTok{ }\KeywordTok{summary}\NormalTok{(model)}
\NormalTok{r2\textless{}{-}}\StringTok{ }\NormalTok{resumo}\OperatorTok{$}\NormalTok{adj.r.squared}
\NormalTok{per\_mod\textless{}{-}}\StringTok{ }\KeywordTok{data.frame}\NormalTok{(erro\_medio, erro\_percmed,r2)}
\NormalTok{per\_mod}
\end{Highlighting}
\end{Shaded}

\begin{verbatim}
##   erro_medio erro_percmed        r2
## 1  0.7663743 0.0002435668 0.9999999
\end{verbatim}

\hypertarget{treinamento-do-modelo-1}{%
\section{Treinamento do modelo}\label{treinamento-do-modelo-1}}

Treinamento do modelo utilizando árvore de regresão.

\begin{Shaded}
\begin{Highlighting}[]
\NormalTok{model2\textless{}{-}}\StringTok{ }\KeywordTok{rpart}\NormalTok{(total}\OperatorTok{\textasciitilde{}}\NormalTok{., }\DataTypeTok{data =}\NormalTok{ treino)}
\end{Highlighting}
\end{Shaded}

\hypertarget{avaliauxe7uxe3o-da-performance-do-modelo-1}{%
\section{Avaliação da performance do
modelo}\label{avaliauxe7uxe3o-da-performance-do-modelo-1}}

Para a avaliação da performance do modelo foi utilizado duas métricas,
sendo elas erro médio absoluto, percentual médio do erro absoluto.
Também foi avaliado a distribuição dos erros dentro de quartis.

\begin{Shaded}
\begin{Highlighting}[]
\NormalTok{predito\textless{}{-}}\KeywordTok{predict}\NormalTok{(model2,teste)}
\NormalTok{per2\textless{}{-}}\StringTok{ }\NormalTok{teste }\OperatorTok{\%\textgreater{}\%}\StringTok{ }\KeywordTok{select}\NormalTok{(}\StringTok{"city"}\NormalTok{, }\StringTok{"total"}\NormalTok{)}\OperatorTok{\%\textgreater{}\%}
\StringTok{  }\KeywordTok{mutate}\NormalTok{(predito) }\OperatorTok{\%\textgreater{}\%}\StringTok{ }\KeywordTok{mutate}\NormalTok{(}\DataTypeTok{erro=}\NormalTok{ total}\OperatorTok{{-}}\NormalTok{predito )}\OperatorTok{\%\textgreater{}\%}
\StringTok{  }\KeywordTok{mutate}\NormalTok{(}\DataTypeTok{erro\_abs=} \KeywordTok{abs}\NormalTok{(erro))}\OperatorTok{\%\textgreater{}\%}\KeywordTok{mutate}\NormalTok{(}\DataTypeTok{erro\_perc=}\NormalTok{ erro}\OperatorTok{/}\NormalTok{total)}\OperatorTok{\%\textgreater{}\%}\StringTok{ }
\StringTok{  }\KeywordTok{mutate}\NormalTok{(}\DataTypeTok{erro\_percabs=} \KeywordTok{abs}\NormalTok{(erro\_perc))}
\NormalTok{per[,}\KeywordTok{c}\NormalTok{(}\DecValTok{4}\OperatorTok{:}\DecValTok{7}\NormalTok{)]\textless{}{-}}\KeywordTok{round}\NormalTok{(per[,}\KeywordTok{c}\NormalTok{(}\DecValTok{4}\OperatorTok{:}\DecValTok{7}\NormalTok{)], }\DecValTok{5}\NormalTok{)}

\CommentTok{\# Calculando o erro medio absoluto e percentual medio}
\NormalTok{erro\_medio2\textless{}{-}}\StringTok{ }\KeywordTok{mean}\NormalTok{(per2}\OperatorTok{$}\NormalTok{erro\_abs)}
\NormalTok{erro\_percmed2\textless{}{-}}\StringTok{ }\KeywordTok{mean}\NormalTok{(per2}\OperatorTok{$}\NormalTok{erro\_percabs)}
\NormalTok{per\_mod2\textless{}{-}}\StringTok{ }\KeywordTok{data.frame}\NormalTok{(erro\_medio2, erro\_percmed2)}
\NormalTok{per\_mod2}
\end{Highlighting}
\end{Shaded}

\begin{verbatim}
##   erro_medio2 erro_percmed2
## 1    1911.981     0.5758513
\end{verbatim}

\begin{Shaded}
\begin{Highlighting}[]
\KeywordTok{summary}\NormalTok{(per2}\OperatorTok{$}\NormalTok{erro\_percabs)}
\end{Highlighting}
\end{Shaded}

\begin{verbatim}
##     Min.  1st Qu.   Median     Mean  3rd Qu.     Max. 
## 0.000784 0.186709 0.380779 0.575851 0.698671 5.846700
\end{verbatim}

\hypertarget{treinamento-do-modelo-2}{%
\section{Treinamento do modelo}\label{treinamento-do-modelo-2}}

Treinamento do modelo utilizando Random Forest.

\begin{Shaded}
\begin{Highlighting}[]
\NormalTok{model3\textless{}{-}}\StringTok{ }\KeywordTok{randomForest}\NormalTok{(total}\OperatorTok{\textasciitilde{}}\NormalTok{., }\DataTypeTok{data =}\NormalTok{ treino,}
                      \DataTypeTok{ntree=} \DecValTok{100}\NormalTok{, }\DataTypeTok{proximity=}\NormalTok{ T)}
\end{Highlighting}
\end{Shaded}

\hypertarget{avaliauxe7uxe3o-da-performance-do-modelo-2}{%
\section{Avaliação da performance do
modelo}\label{avaliauxe7uxe3o-da-performance-do-modelo-2}}

Para a avaliação da performance do modelo foi utilizado duas métricas,
sendo elas erro médio absoluto, percentual médio do erro absoluto.
Também foi avaliado a distribuição dos erros dentro de quartis.

\begin{Shaded}
\begin{Highlighting}[]
\NormalTok{predito\textless{}{-}}\KeywordTok{predict}\NormalTok{(model3,teste)}
\NormalTok{per3\textless{}{-}}\StringTok{ }\NormalTok{teste }\OperatorTok{\%\textgreater{}\%}\StringTok{ }\KeywordTok{select}\NormalTok{(}\StringTok{"city"}\NormalTok{, }\StringTok{"total"}\NormalTok{)}\OperatorTok{\%\textgreater{}\%}
\StringTok{  }\KeywordTok{mutate}\NormalTok{(predito) }\OperatorTok{\%\textgreater{}\%}\StringTok{ }\KeywordTok{mutate}\NormalTok{(}\DataTypeTok{erro=}\NormalTok{ total}\OperatorTok{{-}}\NormalTok{predito )}\OperatorTok{\%\textgreater{}\%}
\StringTok{  }\KeywordTok{mutate}\NormalTok{(}\DataTypeTok{erro\_abs=} \KeywordTok{abs}\NormalTok{(erro))}\OperatorTok{\%\textgreater{}\%}\KeywordTok{mutate}\NormalTok{(}\DataTypeTok{erro\_perc=}\NormalTok{ erro}\OperatorTok{/}\NormalTok{total)}\OperatorTok{\%\textgreater{}\%}\StringTok{ }
\StringTok{  }\KeywordTok{mutate}\NormalTok{(}\DataTypeTok{erro\_percabs=} \KeywordTok{abs}\NormalTok{(erro\_perc))}
\NormalTok{per3[,}\KeywordTok{c}\NormalTok{(}\DecValTok{4}\OperatorTok{:}\DecValTok{7}\NormalTok{)]\textless{}{-}}\KeywordTok{round}\NormalTok{(per3[,}\KeywordTok{c}\NormalTok{(}\DecValTok{4}\OperatorTok{:}\DecValTok{7}\NormalTok{)], }\DecValTok{5}\NormalTok{)}

\CommentTok{\# Calculando o erro medio absoluto e percentual medio}
\NormalTok{erro\_medio3\textless{}{-}}\StringTok{ }\KeywordTok{mean}\NormalTok{(per3}\OperatorTok{$}\NormalTok{erro\_abs)}
\NormalTok{erro\_percmed3\textless{}{-}}\StringTok{ }\KeywordTok{mean}\NormalTok{(per3}\OperatorTok{$}\NormalTok{erro\_percabs)}
\NormalTok{per\_mod3\textless{}{-}}\StringTok{ }\KeywordTok{data.frame}\NormalTok{(erro\_medio3, erro\_percmed3)}

\KeywordTok{summary}\NormalTok{(per3}\OperatorTok{$}\NormalTok{erro\_percabs)}
\end{Highlighting}
\end{Shaded}

\begin{verbatim}
##     Min.  1st Qu.   Median     Mean  3rd Qu.     Max. 
##  0.00000  0.00540  0.01230  0.03504  0.02474 23.05960
\end{verbatim}

\hypertarget{comparauxe7uxe3o-entre-os-truxeas-modelos}{%
\section{Comparação entre os três
modelos}\label{comparauxe7uxe3o-entre-os-truxeas-modelos}}

\begin{Shaded}
\begin{Highlighting}[]
\NormalTok{per\_mod1\textless{}{-}}\StringTok{ }\NormalTok{per\_mod[,}\OperatorTok{{-}}\DecValTok{3}\NormalTok{]}
\NormalTok{Modelo\textless{}{-}}\StringTok{ }\KeywordTok{c}\NormalTok{(}\StringTok{"Regressão Linear"}\NormalTok{, }\StringTok{"Arv. Regressão"}\NormalTok{, }\StringTok{"Random Forest"}\NormalTok{)}
\NormalTok{erro\_med\textless{}{-}}\KeywordTok{c}\NormalTok{(per\_mod}\OperatorTok{$}\NormalTok{erro\_medio, per\_mod2}\OperatorTok{$}\NormalTok{erro\_medio2, per\_mod3}\OperatorTok{$}\NormalTok{erro\_medio3)}
\NormalTok{erro\_permed\textless{}{-}}\KeywordTok{c}\NormalTok{(per\_mod}\OperatorTok{$}\NormalTok{erro\_percmed, per\_mod2}\OperatorTok{$}\NormalTok{erro\_percmed2,}
\NormalTok{                per\_mod3}\OperatorTok{$}\NormalTok{erro\_percmed3)}
\NormalTok{resultado\textless{}{-}}\StringTok{ }\KeywordTok{cbind}\NormalTok{(Modelo, erro\_med, erro\_permed)}
\NormalTok{resultado}
\end{Highlighting}
\end{Shaded}

\begin{verbatim}
##      Modelo             erro_med            erro_permed           
## [1,] "Regressão Linear" "0.766374311740891" "0.000243566801619433"
## [2,] "Arv. Regressão"   "1911.98098579908"  "0.575851318110582"   
## [3,] "Random Forest"    "186.284159012146"  "0.0350397651821862"
\end{verbatim}

A regressão linear apresentou a melhor perfomance de predição para os
valores de aluguel residencial (R²= 0.999). Com um erro médio de 0.766 e
um erro percentual de 0.0002\%. Já Os algoritmos de árvore de regressão
e random forest apresentaram 57 e 3.5\% por cento de erro médio.

\end{document}
